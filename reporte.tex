\documentclass[12pt, letterpaper]{article}
\usepackage[spanish]{babel}
\usepackage[utf8]{inputenc}
\usepackage{xcolor}
\usepackage{graphicx}
\graphicspath{ {./imagenes/} }
\usepackage{amsmath, mathtools}
\usepackage{float}
\usepackage{hyperref}
\title{\textbf{¿Por qué escogí Ciencias de la Computación?}}
\author{Israel Hernández Dorantes}
\date{\today}

\begin{document}
\maketitle

\section{Motivos}
Escogí esta carrera debido a que me gustaba la programación y quería profundizar conceptos sobre el software de una computadora ya que me ha interesado cómo funciona un ordenador. Aunque es cierto que desde pequeño no me interesaba para nada estos temas y no les daba tanta importancia como ahora, pero desde la preparatoria en mi materia de \texttt{Informática} me enseñaron cómo es programar y había quedado encantado con ese mundo y empece a crear por mi cuenta varios programas sencillos como sumar dos números, sacar el factorial de un numero dado por el usuario, etc. Y desde entonces he ido aprendiendo más y más, y poco a poco descubrí que esto era lo que yo quería dedicar mi vida.

\subsection{Aspectos que debo considerar al estudiar esta carrera}
Antes de haber escogido esta carrera tuve varias conferencias, las cuales me indicaron cómo era estudiar \textit{Ciencias de la Computación} en la Facultad de Ciencias, y algunas cosas que me comentaron fueron:
\begin{enumerate}
  \item{Su nivel de exigencia era alto, por lo que debo de esforzarme más que en la preparatoria.}
  \item{Es recomendable contar con una computadora, para poder realizar las prácticas que se me vayan a dejar.}
  \item{El horario iba a estar algo distinto a como yo lo tenía en la preparatoria, por ejemplo que podría tener una clase a las 8 de la mañana y otra hasta las 5, así         que debo de considerar en ir preparado para eso.}
\end{enumerate}
\subsubsection{Mi reacción al quedarme en la carrera que quería}
Antes de que se subieran los resultados sobre el \textit{Pase Reglamentado} yo estaba muy nervioso ya que mi promedio no era tan bueno para poder calificar a una carrera que es de alta demanda, pero durante ese tiempo decidí adentrarme un poco al mundo de la programación, revisando varios temas para no entrar a la carrera sin saber nada, y para que pudiera aprender cada tema estuve en varios sitios:

\[
\begin{tabular}{|c|c|}
  \hline
  \textbf{Temas} & \textbf{Sitios}\\
  \hline
  Programacion en Java & Sitio de cursos en Udemy\\
  \hline
  Programacion en C & Tutoriales de Youtube\\
  \hline
  Creacion de aplicaciones moviles para iOS & Sitio de cursos Udemy\\
  \hline
  Uso basico de Linux & Tutoriales de Youtube\\
  \hline
  Crear paginas web & Sitio de cursos en Udemy\\
  \hline
\end{tabular}
\]

Link de página de cursos: \url{https://www.udemy.com/}
\\
\\
Después de haber aprendido algo, a la hora de la publicación de los resultados, quedé sorprendido ya que si me había quedado en mi primer opción, aunque no me lo creía, y estaba pensando que lo que había aprendido en las vacacione si me iba a servir para la carrera y me puse muy feliz.
\newpage
    
\section{¿Qué especialización me gustaría estudiar}
Últimamente desde que cree mi primera aplicación para una un iPhone y mi primera página web, quedé fascinado en cuanto al desrrollo de ese software, por lo que he considerado mucho en especializarme en esa área, la cual podría ser:
\begin{itemize}
\item Desarrollo de Software
\item Desarrollo de Aplicaciones Móviles
\item Desarrollo Web
\end{itemize}
\begin{figure}[h]
  \centering
  \includegraphics[scale=.25]{imagenDesarrolloAppWeb}
  \caption{Icono de Desarrollo de Software}
  \label{fig:icono}
\end{figure}

\begin{figure}[H]
  \raggedleft
  \includegraphics[scale=.22]{desarrolloApps}\\
\end{figure}

\begin{figure}[H]
  \raggedright
  \includegraphics[scale=.25]{desarrolloWeb}\\
\end{figure}


\section{Expresiones Matemáticas}
\[
x^{2\alpha} - 1 = y_{ij} + y_{ij}\\
\]
\\
\\
\\
\[
F = G \left(\frac{m_{1}m_{2}}{r^2}\right)\\
\]
\\
\\
\\
\begin{equation*}
  \begin{pmatrix}
    1 & 2 & 3\\
    a & b & c\\
    x & y & z
  \end{pmatrix}
\end{equation*}
\\
\\
\\
\begin{equation*}
\begin{pmatrix}
  n\\
  k
\end{pmatrix} = \frac{n!}{k!(n-k)!}
\end{equation*}
\\
\\
\\
\[
f(x) = \displaystyle\sum_{i=0}^n \frac{a_i}{1+x}
\]
\newpage

%Bibliografía
\begin{thebibliography}{x}

\bibitem{knuthwebsite}
  Trejos, O.I. (1999)
  \textit{La Esencia de la Programación}. Colombia: Papiro

\bibitem{knuthwebsite}
  Urias, E. (2020)
  \textit{¿Qué es el desarrollo de aplicaciones móviles}. InvidGroup. Recuperado el 9 de Octubre del 2020 de \url{https://invidgroup.com/es/que-es-el-desarrollo-de-aplicaciones-moviles/}
\end{thebibliography}


\end{document}
